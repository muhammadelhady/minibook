\documentclass [12pt]{book}
\usepackage[utf8]{inputenc}
\usepackage{graphicx}
\graphicspath{ {/images} }
\author{Mohamed El-hady}

\begin{document}
\includegraphics[scale=1.75,width=\textwidth]{cover}

\title{Cloud Computing }
	\maketitle
	 \addcontentsline{toc}{chapter}{Preface}
	 \tableofcontents 
	\chapter*{Preface}
	with the growth of the new technology on the internet all the companies works on making all their apps available on the internet with a new technique making them available in very wide range all over the world this technique called the cloud computing.
	\newline
	cloud computing is kind of computing based on internet to providing sharing resources between users based on their demand it`s provide computing resources and storage to users and enterprises with change of capabilities to process and store the data.
	\newline
	the data centers may be too far away from the users or enterprise that use the cloud could be countries or even across all the world and they just need a internet connection to make access on it and use it to finish their tasks on it in very easy and efficient way.
	\newline
	the growth in cloud computing field is very fast, well widespread and big competitive in that field led to the low coast resources and storage and it's become highly demand in the global market cause it`s avoid the enterprise infrastructure costs ( e.g data centers and servers )    and enable the enterprise to just focus on their core bussiness without wasting their time and resources on infrastructure.
	
	

	\chapter{Why Cloud Computing }
	
	why we move to cloud computing actually there's a lot of benefits from moving to cloud computing and now we will discuss the most reasons for moving to cloud computing 
	 \newline
	 \newline
	 \newline
	
	
	flexibility, Cloud-based services are ideal for businesses with huge growing or fluctuating bandwidth demands. you simply can increase your cloud computing capacity as the increase of your needs, drawing on the remote server of the service. and also if you need to down your scale again, the service flexibility is baked into the concept of cloud computing . This level of agility can give businesses using cloud computing a real advantage over competitors – it’s not surprising that companies rank ‘operational agility’ adopt the cloud computing.
	\newline
	\newline
	\newline
	
	Disaster recovery,  all sizes Businesses  must be investing in fast recovery from  disasters, but the small business have a lack in the required  expertise and Cash, this is often  ideal more than real. CloudComputing helping more companies buck that trend. According to Aberdeen Group, small size businesses are twice than larger companies to  implemented cloudbased recovery and backup to save time, avoid large up-front investment and  third-party expertise included in the deal.
	\newline
	\newline
	\newline
	
	Work from anywhere, using cloud computing, you just need an internet connection to be at work. And most of  cloud services provide mobile apps, you can use any device to continue in your work.
	
	thats led to, Businesses can provide flexible working environment to their employees to can enjoy the work and life balance without productivity taking a hit. a study reported that 42% of employees would swap a part of their salary for the  telecommute.
	\newline
	\newline
	\newline
	
	
	Security, Lost device is a billion dollar problem for business. And maybe greater than the loss of an precious device is the loss of the inside sensitive data. Cloud computing provide you more security when things like this happens. Because your work data stored in the database of the cloud, you can retrive it no matter what happend to the device. also you could remotely erase data from lost device so it saved from wrong hands.
	
	
	\chapter{Moving to the Cloud}
	\chapter{Cloud Computing Arc}
	\section{Fundmentals cloud arc}
    \section{Advanced cloud arc}
    \chapter{security of cloud computing}
    
    
    
    
    \chapter{History of cloud computing}
    
    
    when we say cloud computing you think that cloud computing is from 21st century ,but that's not true at all the cloud computing concept existed in 1950s with mainframes computers where multiple users able to access on the same central computer through dumb terminals which just one of many views to access to the mainframe computer, cause that was not practice to every enterprise to buy or maintain a computer for every one of the company employees 
  \newline
  \newline
  \newline
    in 1970 virtual machines concept was invented and using it become easy to execute more than one operating system concurrency at the same time and the same hardware with its own isolated environment, virtual can execute inside one hardware and can run completely different OS at the same time 
    \newline
    \newline
    \newline
    In 1990s the companies of  telecommunications , who previously offered  dedicated point to point lines, start to offer VPN  (virtual private network)  services with same quality of service, but with lower costs. By switching traffic as they saw fit to balance server , now they can get benfit of the overall bandwidth more effective.They start use the cloud symbol to denote the demarcation point from what the provider was responsible for to  what users were responsible for. Cloud computing increase this boundary to can  cover allover the servers as well as the  infrastructure of network. As computers became more popular, technologists and scientists discoverd ways to make large scale power of computing  available to big number of users through sharing. They used algorithms to optimize the network  infrastructure and applications to prioritize CPUs and increase efficiency for end users
    \newline
    \newline
    \newline
  
    
    
   
    
    Since 2000, cloud computing has come into existence. In early 2008, NASA's OpenNebula, enhanced in the RESERVOIR European Commission-funded project, became the first open-source software for deploying private and hybrid clouds, and for the federation of clouds. In the same year, efforts were focused on providing quality of service guarantees (as required by real-time interactive applications) to cloud-based infrastructures, in the framework of the IRMOS European Commission-funded project, resulting in a real-time cloud environment.  By mid-2008, Gartner saw an opportunity for cloud computing "to shape the relationship among consumers of IT services, those who use IT services and those who sell them" and observed that "organizations are switching from company-owned hardware and software assets to per-use service-based models" so that the "projected shift to computing ... will result in dramatic growth in IT products in some areas and significant reductions in other areas.
    
   


\end{document}